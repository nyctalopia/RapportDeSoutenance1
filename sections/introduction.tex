\section{Introduction}
\setlength{\parindent}{5ex}
Dans le cadre du projet de S2 à EPITA, nous sommes poussés à mettre en pratique les différentes connaissances acquises en Travaux Pratiques et en cours dans la réalisation d'un projet en groupe.
Le projet de notre studio gameHUB a été la création d'un jeu d'horreur: \emph{Nyctalopia}.


A l'arrivée de cette première soutenance, nous nous somme concentré sur l'aspect technique du jeu, le ``backend'', toute la partie que le ou les joueurs ne verront pas, ce qui nous permettra ensuite d'avancer plus rapidement dans notre projet, surtout dans la partie plus visuelle.

C'est ainsi qu'on a réussi à mettre en place le code fondamental du jeu: tout ce qui concerne le déplacement du joueur et de la caméra, le multijoueur et surtout le menu principal du jeu avec ses différentes options.

Nous nous sommes souciés aussi de l'aspect de notre jeu face au public pour cette première soutenance, notamment par la mise en fonctionnement du site web \emph{Nyctalopia} qui est déjà complètement fonctionnel. Ceci et les logos que nous avons crée autant pour le jeu que pour le studio, nous forge déjà une identité d'un point de vue externe.

Nous présentons, dans ce rapport de soutenance, tout le progrès réalisé dans notre projet dans cette première période de travail, comment les tâches ont été réparties, et les objectifs que l'on envisage d'atteindre pour la prochaine soutenance.

\vfill
\noindent\makebox[\linewidth]{\rule{.8\paperwidth}{.6pt}}\\[0.2cm]
EPITA Toulouse - Projet S2 - 2022 \hfill Nyctalopia - gameHUB
\noindent\makebox[\linewidth]{\rule{.8\paperwidth}{.6pt}}

\newpage